\chapter{Revisão Bibliográfica}
\label{revisao}

\section{Python}

\section{Sobre a Transmissão dos Arquivos}
A primeira etapa do desenvolvimento do programa foi definir o método e protocolo que seriam usados na hora de transmitir arquivos entre os computadores.
Após alguma pesquisa, quatro alternativas foram consideradas: 
\begin{itemize}
  \item \textbf{Sockets via TCP}\cite{cerf2005protocol}, que utilizaria a biblioteca padrão \texttt{socket} do Python\cite{pythonsockets} para transmitir os dados dos arquivos a nível de camada de transporte. A implementação em Python é relativamente simples, mas requer muita atenção porque arquivos maiores têm que ser divididos em blocos para depois serem re-montados.
  \item \textbf{Secure Copy} (SCP), método que utiliza os mesmos padrões de segurança do protocolo SSH (\textit{secure shell}) para garantir a segurança e confidencialidade dos dados. A maneira mais simples de fazer isso em Python seria invocando um comando da linha de comando, que não dá muito controle sobre a operação de dentro do programa.
  \item \textbf{File Transfer Protocol}\cite{rfc959} (FTP) é um protocolo da camada de aplicação específico para o envio de arquivos. Contudo, ele se mostra mais eficaz em situações onde há servidor e clientes bem definidos: ou seja, um servidor que hospeda arquivos e clientes que podem acessar e alterar os arquivos do servidor.
  \item \textbf{Hypertext Transfer Protocol}\cite{rfc2068} (HTTP) é o protocolo padrão de envio de dados na Internet. É fácil de utilizar, mas os arquivos ficam temporariamente abertos a qualquer membro da rede.
\end{itemize}

[TO-DO: Remover referências ao WiSync]
Para o WiSync, o HTTP foi escolhido para realizar a troca de arquivos.
Isso se deve principalmente à facilidade de implementar essa troca, e também ao fato que dois computadores podem rapidamente trocar seus papéis de cliente e servidor enquanto transmitem arquivos entre si.
Como o objetivo do WiSync é ser usado em redes locais privadas, a abertura temporária dos arquivos na rede não é uma grande preocupação.

\section {JSON}
O \textit{JavaScript Object Notation}, ou JSON\cite{json}, é um formato de armazenamento e troca de dados baseado na notação de objetos do JavaScript. Com o JSON, é possível armazenar objetos de um programa em um arquivo de texto, utilizando os tipos dictionary, array, string, number e boolean do JavaScript.
O JSON foi escolhido por sua simples integração com Python e por ser um formato facilmente legível por um humano, facilitando a verificação dos dados durante os testes.
Contudo, essa facilidade de leitura também se mostra uma desvantagem, já que é possível alterar os dados armazenados por fora do WiSync e então gerar resultados inesperados.