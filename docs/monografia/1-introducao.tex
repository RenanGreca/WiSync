\chapter{Introdução}
\label{introducao}
Com o barateamento e a popularização de microcomputadores, é comum hoje um escritório, uma família ou até mesmo um indivíduo ter diversos computadores pessoais à sua disposição.
Portanto, torna-se útil e às vezes necessário existir uma maneira de manter esses computadores ``conversando'' entre si, trocando informações e arquivos para que o usuário possa mudar de um para outro de forma coesa.
Além disso, computadores portáteis (i.e., \textit{laptops}) são mais vulneráveis do que computadores domésticos a roubos ou danos físicos, além de outros fatores que podem comprometer a integridade das informações neles contidas, o que torna necessária a realização de \textit{backups} frequentes.

Atualmente existem alguns programas disponíveis na Internet que fazem operações desse tipo, mas em contextos um pouco diferentes.

Entre produtos comerciais, há diversos serviços de armazenamento em nuvem\footnote{Com ``serviço em nuvem'', entende-se que o serviço funciona através de um servidor (ou um conjunto deles) na Internet cuja localização não é necessariamente conhecida pelo usuário.\cite{mell2011nist}} que permitem que o usuário sincronize seus arquivos com um servidor na nuvem e com outros dispositivos, caso seja instalado um aplicativo em cada dispositivo. 
[TO-DO: referência sobre cloud services?]
Exemplos desses produtos incluem Dropbox \cite{dropbox}, Box \cite{box}, Google Drive \cite{googledrive}, Apple iCloud Drive \cite{icloud} e Microsoft OneDrive \cite{onedrive}.
A principal diferença desses produtos ao atual projeto é a existência de um servidor mantendo os arquivos na nuvem.
A vantagem dessa decisão é que o usuário pode acessar seus arquivos facilmente de outros dispositivos como celulares e tablets \-- contudo, para a sincronização ocorrer é necessário que os computadores clientes estejam conectados à Internet e todos esses serviços impõem limites em bytes na quantidade de arquivos que podem ser sincronizados.

Outra categoria de programas que realizam operações semelhantes são os programas de \textit{backup}.
Um exemplo desses programas é o FreeFileSync \cite{freefilesync}, solução \textit{open-source} disponível na web.
Dados um par de diretórios A e B, programas de \textit{backup} normalmente possuem três funcionalidades:
\begin{itemize}
    \item Sincronização de dois sentidos, que torna A e B idênticos copiando as modificações de um para o outro e vice-versa;
    \item Sincronização de espelho, que faz o diretório B ser um clone do diretório A, incluindo arquivos removidos; e
    \item Sincronização de atualização, que atualiza B com os arquivos novos ou modificados de A, mas não remove arquivos excluídos em A.
\end{itemize}

Para o propósito desta monografia, o projeto irá focar no primeiro tipo de sincronização, a de dois sentidos.
Os outros dois tipos também são interessantes e possivelmente serão adicionados ao escopo futuramente.

Enquanto o FreeFileSync sincroniza dois diretórios no mesmo computador e os serviços em nuvem se beneficiam de um servidor remoto que gerencia a sincronização, o objetivo deste projeto é desenvolver um programa que funcione de maneira local, ou seja, fazendo dois computadores interagirem pela rede sem um servidor no meio.
Comparado às soluções em nuvem, isso permite que o programa seja rodado sem uma conexão à Internet, não impõe limites de dados e também resulta numa velocidade maior na transferência dos arquivos.
Contudo, requer que ambos computadores estejam conectados à mesma rede e torna a sincronização em si um pouco mais complicada.

Programas que dependem da comunicação em rede são naturalmente desafiadores, pois eles dependem muito de fatores externos para funcionarem como esperado.
No caso deste projeto, duas instâncias do programa estarão rodando simultaneamente numa rede e devem se encontrar e se comunicar.
[TO-DO: Elaborar sobre os desafios do projeto]

\section{Proposta}
Para trabalhar em diversos computadores ou manter cópias seguras de arquivos de forma conveniente, o presente projeto propõe um programa que, através de uma rede local (daqui em diante referida como LAN, de \textit{Local Area Network}), compare diretórios em dois ou mais computadores distintos e realize a sincronização dos mesmos.
Com sincronização, queremos dizer que o programa irá, para cada instância rodando:
\begin{itemize}
    \item Copiar arquivos novos (adicionados desde a última execução) às outras instâncias;
    \item Apagar arquivos removidos nas outras instâncias; e
    \item Copiar alterações aos arquivos às outras instâncias, incluindo arquivos renomeados.
\end{itemize}