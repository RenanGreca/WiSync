\chapter{Resultados}
\label{resultados}

\section{Organização do Programa}
O WiSync é composto por três arquivos: \texttt{wisync.py}, \texttt{winet.py} e \texttt{wifiles.py}. Abaixo estão as funcionalidades de cada um:
\begin{itemize}
  \item \texttt{wisync.py}: Arquivo principal do projeto, responsável por ler os parâmetros da linha de comando e controlar a execução do processo.
  \item \texttt{winet.py}: Contém a classe WiNet, que inclui os métodos e atributos necessários para fazer as partes em rede do programa, como hospedar e receber arquivos.
  \item \texttt{wifiles.py}: Contém a classe WiFiles, que inclui os métodos e atributos necessários fazer as partes que lidam com o sistema de arquivos do programa, como ler e comparar diretórios.
\end{itemize}

\subsection{Parâmetros de Execução}

Para rodar o WiSync, utiliza-se o seguinte comando:
\begin{verbatim}
python wisync.py [-h] -d DIRECTORY [-n HOSTNAME] [-s]
\end{verbatim}

Os argumentos são os seguintes:
\begin{itemize}
  \item \texttt{-h, --help} Mostra o texto de ajuda do programa.
  \item \texttt{-d DIRECTORY, --directory DIRECTORY} Caminho para o diretório a ser sincronizado (de preferência, o caminho completo)
  \item \texttt{-n HOSTNAME, --hostname HOSTNAME} Nome de rede ou endereço IP do outro computador. Esse argumento é opcional e pode ser preenchido tanto pelo endereço IP do outro computador quanto pelo nome de rede. Por exemplo, \texttt{-n SgtPepper.local} e \texttt{-n 192.168.1.110} têm o mesmo efeito. Caso o argumento não esteja presente, o programa procura outras instâncias do WiSync na rede local.
  \item \texttt{-s, --server} Modo servidor. Se definido, esta instância será o ``servidor'', hospedando seus arquivos antes da instância remota.
\end{itemize}
