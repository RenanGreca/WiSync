\chapter{Implementa��o}
\label{Implementacao}
Neste cap�tulo, ser� descrita a implementa��o do \textbf{WiSync} para este trabalho.

A vers�o do \textbf{WiSync} entregue com este texto foi escrita em Python \cite{python}, vers�o 2.7, linguagem de programa��o interpretada originalmente lan�ada em 1991. Python foi escolhida por sua facilidade na implementa��o e extensa disponibilidade de bibliotecas de c�digo aberto.

Originalmente, havia como objetivo tornar o programa compat�vel com os tr�s sistemas operacionais mais comuns do mundo: Microsoft Windows, Apple OS X e Linux. Contudo, devido a diferen�as inerentes na forma como o Windows funciona, optou-se por manter apenas compatibilidade com OS X e Linux. Para tal, foram usados os seguintes computadores de teste:

\begin{center}
  \begin{tabular}{ l | l | l }
    \hline
    \textbf{Nome} & ``SgtPepper'' & ``Packard'' \\ \hline \hline
    Sistema Operacional & OS X 10.10 & Linux Mint 17 \\ \hline
    Processador & Intel Core i5-4308U & Intel Core i7-2600 \\ \hline
    RAM & 8GB DDR3L & 8GB DDR3 \\ \hline
    Conectividade & Wi-fi 802.11n 5GHz & Cabo Ethernet \\ \hline
    IP local & 192.168.1.110 & 192.168.1.132 \\
    \hline
  \end{tabular}
\end{center}

\section{Organiza��o do Programa}
Na vers�o atual, o WiSync � composto por tr�s arquivos: \texttt{wisync.py}, \texttt{winet.py} e \texttt{wifiles.py}. Abaixo est�o as funcionalidades de cada um:
\begin{itemize}
	\item \texttt{wisync.py}: Arquivo principal do projeto, respons�vel por ler os par�metros da linha de comando e controlar a execu��o do processo.
	\item \texttt{winet.py}: Cont�m a classe WiNet, que inclui os m�todos e atributos necess�rios para fazer as partes em rede do programa, como hospedar e receber arquivos.
	\item \texttt{wifiles.py}: Cont�m a classe WiFiles, que inclui os m�todos e atributos necess�rios fazer as partes que lidam com o sistema de arquivos do programa, como ler e comparar diret�rios.
\end{itemize}

Para executar o \textbf{WiSync}, s�o necess�rias algumas bibliotecas padr�o do Python: \texttt{os}, \texttt{sys}, \texttt{argparse}, \texttt{time}, \texttt{json}, \texttt{datetime}. Tamb�m � usada uma vers�o modificada do programa \texttt{woof.py} \cite{woof} (distribu�da sob a licen�a GNU General Public License), que � usada na hora de transmitir os arquivos entre os computadores.

\section{Sobre a Transmiss�o dos Arquivos}
A primeira etapa do desenvolvimento do programa foi definir o m�todo e protocolo que seriam usados na hora de transmitir arquivos entre os computadores. Ap�s alguma pesquisa, quatro alternativas foram consideradas: sockets via TCP, FTP, SCP e HTTP.